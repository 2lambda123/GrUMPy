\documentclass{beamer}

\usepackage{graphics}

\usetheme{Boadilla}


\newcommand{\ignore}[1]{}
\newcommand{\Real}{\mathbf{R}}

\title[Visualization]{Visualization of Branch-and-bound Algorithms}

\author[Ted]{Ted Ralphs}

\institute[Lehigh University]{Lehigh University}
\date{\today} 
\begin{document}

%=========================== Title page =======================================
\begin{frame}
\titlepage
\end{frame}

%=========================== Collaborators ====================================
\begin{frame}
\frametitle{Collaborators}
\begin{center}
Aykut Bulut\\
and others
\end{center}
\end{frame}

%=========================== slide-1 ==========================================
\begin{frame}
\frametitle{Outline}
\begin{itemize}
\item Reasons to monitor the progress of branch-and-bound
\item Current measures
\item Our approach
\item Visualization tools
\item Current and future work
\end{itemize}
\end{frame}

%=========================== slide-2 ==========================================
\begin{frame}
\frametitle{Why monitor the progress of B\&B algorithms?}

\uncover<+->{}
\begin{itemize}[<+->]
\item How good is the best solution so far?
\item How much longer do we have to wait until the algorithm terminates?
\item How likely is it that a better solution will be found, and how much better will it be?
\item Should we change any algorithm strategies? (branching, node selection,...) 
\end{itemize}
\end{frame}

%=========================== slide-3 ==========================================
\begin{frame}
\frametitle{How do we monitor the progress of B\&B now?}

\begin{itemize}[<+->]
\item Most commercial and open-source solvers can report:
    \begin{itemize}
    \item optimality gap
    \item number of active nodes
    \item some internal measures for guiding the algorithm
    \end{itemize} 
\item Each of these methods have some strengths and weaknesses 
\end{itemize}
\end{frame}

%=========================== slide-4 ==========================================
\begin{frame}
\frametitle{Current methods: optimality gap}

\begin{center}
optimality gap picture
\end{center}

\begin{itemize}[<+->]
\item Strength: guarantee on quality of solution
\item Strength: nonincreasing
\item \alert {Weakness: may remain constant for long periods, then drop suddenly } 
\end{itemize}
\end{frame}

%=========================== slide-5 ==========================================
\begin{frame}
\frametitle{Current methods: number of active nodes}

\begin{center}
number of activenodes picture
\end{center}

\begin{itemize}[<+->]
\item Strength: some sense of "work remaining"
\item \alert {Weakness: may go up and down}
\item \alert {Weakness: each active node counts equally } 
\end{itemize}
\end{frame}

%=========================== slide-6 ==========================================
\begin{frame}
\frametitle{Prior work and inspiration}

\begin{itemize}[<+->]
\item VBCTool (Diehl, J\"unger, Kupke, Leipart)
\item Images on the TSP website (www.tsp.gatech.edu)
\item XPRESS-MP
\item Cornu\'ejols, Karamanov, Li (2006)
\end{itemize}
\end{frame}


%=========================== slide-7 ==========================================
\begin{frame}
\frametitle{Key Observations}

\begin{enumerate}[<+->]
\item B\&B algorithms generate lots of data during the solution procedure
    \begin{itemize}[<+->]
    \item Number of nodes
    \item For each node:
        \begin{itemize}         
        \item LP Bound
        \item integer infeasibility
        \item history/position in tree (e.g. depth and parent) 
        \end{itemize}
    \end{itemize} 
\item Current methods use only a small amount of this data
\item Most prior work only considers one type of information at a time
\item \alert {Develop tools that consider as much data as possible!} 
\end{enumerate}
\end{frame}

%=========================== slide-8 ==========================================
\begin{frame}
\frametitle{Approach}

\begin{itemize}[<+->]
\item Used open-source solvers GLPK, SYMPHONY, and CBC
\item Modified solver codes to output information when nodes are added and
processed
\item Wrote new code to create visual representations of the data by parsing
the output file
    \begin{itemize}[<+->]
    \item Output file can be processed at any point during the solving process 
    \item Parsing is done with Perl; images are created with Gnuplot
    \end{itemize}
\end{itemize}
\end{frame}

%=========================== slide-9 ==========================================
\begin{frame}
\frametitle{Example of output from solver}

%\begin{verbatim}
\# CBC\\
0.040003 heuristic -28.000000\\
2.692169 branched 0 -1 N -39.248099 16 0.169729\\
2.692169 pregnant 2 0 R -39.248063 14 105.991922\\
2.708170 pregnant 3 0 L -38.939929 6 0.105246\\
2.764173 pregnant 5 2 R -39.244862 12 49.115388\\
2.764173 branched 2 0 R -39.248063 14 105.991922\\
%\end{verbatim}
\end{frame}

%=========================== slide-10 =========================================
\begin{frame}
\frametitle{Visual Representations}

\begin{itemize}
\item Visual representations:
    \begin{itemize}[<+->]    
    \item Histogram of active node LP bounds
    \item Scatter plot of active node LP bounds \& integer infeasibility
    \item Incumbent node history in scatter plot
    \item B\&B trees showing the LP bound of each node
    \end{itemize}
\end{itemize}
\end{frame}

%=========================== slide-11 =========================================
\begin{frame}
\frametitle{Visualization tools: Histogram of active node LP bounds}

\begin{center}
histogram of active node lp bounds
\end{center}

\begin{itemize}
\item Horizontal axis is the LP bound
\item Vertical axis is number of active nodes
\item Green vertical line shows the current incumbent value and the blue one
shows the overall LP bound
\end{itemize}
\end{frame}

%=========================== slide-12 =========================================
\begin{frame}
\frametitle{Example histogram series 1: l152lav (MIPLIB 2003)}

\begin{center}
%\only<+>{\includegraphics[width=10cm]{pictures/l152lav/objectives000.png}}%
%\only<+>{\includegraphics[width=10cm]{pictures/l152lav/objectives001.png}}%
histogram series
\end{center}
\end{frame}

%=========================== slide-13 =========================================
\begin{frame}
\frametitle{Example histogram series 2: swath (MIPLIB 2003)}

\begin{center}
%\only<+>{\includegraphics[width=10cm]{pictures/swath/objectives000.png}}%
%\only<+>{\includegraphics[width=10cm]{pictures/swath/objectives001.png}}%
histogram series 2
\end{center}
\end{frame}

%=========================== slide-14 =========================================
\begin{frame}
\frametitle{Visualization tools: Scatter plot of active node LP bounds \& integer infeasibility}

\begin{center}
scatterplot
\end{center}

\begin{itemize}
\item Horizontal axis is the integer infeasibility
\item Vertical axis is the LP bound
\item Green horizontal line is the current incumbent value
\end{itemize}
\end{frame}

%=========================== slide-15 =========================================
\begin{frame}
\frametitle{Example scatter plot series 1: swath (MIPLIB 2003)}

\begin{center}
%\only<+>{\includegraphics[width=10cm]{pictures/swath/scatter000.png}}%
%\only<+>{\includegraphics[width=10cm]{pictures/swath/scatter001.png}}%
scatterplot series
\end{center}
\end{frame}

%\begin{frame}
%\frametitle{Patterns in integer infeasibility: SYMPHONY}
%\begin{center}
%\only<+>{\includegraphics[width=10cm]{pictures/mod008scatterplot003.png}}%
%\only<+>{\includegraphics[width=10cm]{pictures/mod008scatterplot004.png}}%
%\end{center}
%\end{frame}

%=========================== slide-16 =========================================
\begin{frame}
\frametitle{Visualization tools: Incumbent node history in scatter plot}

\begin{center}
incumbent history plot
\end{center}
\begin{itemize}
\item Horizontal axis is the integer infeasibility
\item Vertical axis is the LP bound
\item Green line shows ancestors of the incumbent node
\end{itemize}
\end{frame}

%=========================== slide-17 =========================================
\begin{frame}
\frametitle{Example incumbent node history series 1: l152lav (MIPLIB 2003)}

\begin{center}
%\only<+>{\includegraphics[width=10cm]{pictures/l152lav/incdata000.png}}%
%\only<+>{\includegraphics[width=10cm]{pictures/l152lav/incdata001.png}}%
incumbent series
\end{center}
\end{frame}

%=========================== slide-18 =========================================
%\begin{frame}
%\frametitle{Example incumbent node history series 2: liu (MIPLIB 2003)}

%\begin{center}
%\includegraphics[width=10cm]{pictures/liu/incdata000.png}
%\end{center}
%\end{frame}

%=========================== slide-19 =========================================
\begin{frame}
\frametitle{Visualization tools: B\&B trees}

\begin{center}
tree image
\end{center}
\begin{itemize}
\item Vertical axis is the LP bound
\item Nodes are horizontally positioned to make the pictures more readable
\item Alternatively, horizontal positions may be fixed based on position in the
tree
\end{itemize}
\end{frame}

%=========================== slide-20 =========================================
\begin{frame}
\frametitle{Visualization tools: B\&B trees}

\begin{center}
tree image
\end{center}
\begin{itemize}
\item Node color legend:
    \begin{itemize}
    \item green: branched
    \item yellow: candidate or pregnant
    \item red: fathomed
    \item blue: infeasible
    \end{itemize} 
\end{itemize}
\end{frame}

%=========================== slide-21 =========================================
\begin{frame}
\frametitle{Example B\&B trees}

\begin{center}
%\only<+>{\includegraphics[width=10cm]{pictures/natasa/tree000.png}}%
%\only<+>{\includegraphics[width=10cm]{pictures/natasa/tree001.png}}%
tree series
\end{center}
\end{frame}

%=========================== slide-22 =========================================
\begin{frame}
\frametitle{Example B\&B tree series 1: l152lav (MIPLIB 2003)}

\begin{center}
%\only<+>{\includegraphics[width=10cm]{pictures/l152lav/tree000.png}}%
%\only<+>{\includegraphics[width=10cm]{pictures/l152lav/tree001.png}}%
tree series
\end{center}
\end{frame}

%\begin{frame}
%\frametitle{Example B\&B tree series 2: liu (MIPLIB 2003)}
%\begin{center}
%\only<+>{\includegraphics[width=10cm]{pictures/liu/tree000.png}}%
%\only<+>{\includegraphics[width=10cm]{pictures/liu/tree001.png}}%
%\only<+>{\includegraphics[width=10cm]{pictures/liu/tree002.png}}%
%\only<+>{\includegraphics[width=10cm]{pictures/liu/tree003.png}}%
%\only<+>{\includegraphics[width=10cm]{pictures/liu/tree004.png}}%
%\only<+>{\includegraphics[width=10cm]{pictures/liu/tree005.png}}%
%\only<+>{\includegraphics[width=10cm]{pictures/liu/tree006.png}}%
%\only<+>{\includegraphics[width=10cm]{pictures/liu/tree007.png}}%
%\end{center}
%\end{frame}


%\begin{frame}
%\frametitle{Example B\&B tree series 3}
%\begin{center}
%\only<+>{\includegraphics[width=10cm]{pictures/natasa/tree000.png}}%
%\only<+>{\includegraphics[width=10cm]{pictures/natasa/tree001.png}}%
%\only<+>{\includegraphics[width=10cm]{pictures/natasa/tree002.png}}%
%\only<+>{\includegraphics[width=10cm]{pictures/natasa/tree003.png}}%
%\only<+>{\includegraphics[width=10cm]{pictures/natasa/tree004.png}}%
%\only<+>{\includegraphics[width=10cm]{pictures/natasa/tree005.png}}%
%\only<+>{\includegraphics[width=10cm]{pictures/natasa/tree006.png}}%
%\only<+>{\includegraphics[width=10cm]{pictures/natasa/tree007.png}}%
%\only<+>{\includegraphics[width=10cm]{pictures/natasa/tree008.png}}%
%\only<+>{\includegraphics[width=10cm]{pictures/natasa/tree009.png}}%
%\only<+>{\includegraphics[width=10cm]{pictures/natasa/tree010.png}}%
%\end{center}
%\end{frame}

%=========================== slide-23 =========================================
\begin{frame}
\frametitle{Imagined uses}

\begin{itemize}[<+->]
\item Predict time to completion (through experienced user or automated) 
\item Quickly determine what makes an instance hard
\item Identify key LP solutions that should be investigated
\end{itemize}
\end{frame}

%=========================== slide-24 =========================================
\begin{frame}
\frametitle{Summary and Current Efforts}
\begin{itemize}[<+->]
\item Our code and graphs provide a variety of visual information for users of
branch-and-bound algorithms.
\item A paper has been submitted.  A pre-print is available at Optimization
Online.
\item The code has been submitted to COIN-OR.  At present, available at
www.engr.pitt.edu/hunsaker/
\item Future work: estimate likelihood of better integer solutions
\item Can other information be extracted: recommended node selection strategy
or cuts?
\item What will other researchers do with the tools?
\end{itemize}
\end{frame}

\end{document}
